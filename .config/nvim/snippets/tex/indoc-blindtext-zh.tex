壬戌之秋,七月既望,蘇子與客泛舟遊於赤壁之下。清風徐來,水波不興。舉酒屬客,誦明月之詩,歌窈窕之章。少焉,月出於東山之上,徘徊於斗牛之間,白露橫江,水光接天,縱一葦之所如,陵萬頃之茫然。浩浩乎如慿虛御風,而不知其所止;飄飄乎如遺世獨立,羽化而登僊。

於是飲酒樂甚,扣舷而歌之。歌曰:「桂棹兮蘭槳,擊空眀兮泝流光。渺渺兮予懷,望美人兮天一方。」客有吹洞簫者,倚歌而和之,其聲嗚嗚然,如怨、如慕、如泣、如訴,餘音嫋嫋,不絕如縷。舞幽壑之潛蛟,泣孤舟之嫠婦。

蘇子愀然,正襟危坐,而問客曰:「何為其然也?」

客曰:「『月眀星稀,烏鵲南飛』,此非曹孟德之詩乎?西望夏口,東望武昌,山川相繆,鬱乎蒼蒼,此非孟德之困於周郎者乎?方其破荊州,下江陵,順流而東也,舳艫千里,旌旗蔽空,釃酒臨江,橫槊賦詩,固一世之雄也,而今安在哉?況吾與子漁樵於江渚之上,侶魚蝦而友麋鹿;駕一葉之扁舟,舉匏罇以相屬。寄蜉蝣於天地,渺滄海之一粟。哀吾生之須臾,羨長江之無窮。挾飛仙以遨遊,抱眀月而長終。知不可乎驟得,託遺響於悲風。」

蘇子曰:「客亦知夫水與月乎?逝者如斯,而未嘗往也;贏者如彼,而卒莫消長也,蓋將自其變者而觀之,則天地曾不能以一瞬;自其不變者而觀之,則物與我皆無盡也,而又何羨乎?且夫天地之閒,物各有主,苟非吾之所有,雖一毫而莫取。惟江上之清風,與山閒之眀月,耳得之而為聲,目遇之而成色,取之無禁,用之不竭,是造物者之無盡藏也,而吾與子之所共食。」

客喜而笑,洗盞更酌。肴核既盡,杯盤狼籍,相與枕藉乎舟中,不知東方之既白。
